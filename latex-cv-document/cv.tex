%% start of file `template.tex'.
%% Copyright 2006-2013 Xavier Danaux (xdanaux@gmail.com).
%% https://www.ctan.org/pkg/moderncv?lang=en


\documentclass[10pt,a4paper,roman]{moderncv}
\moderncvstyle{casual}                             % style options are 'casual'
                                                    % (default), 'classic', 'oldstyle' and
                                                    % 'banking'
%% Choosing color
\moderncvcolor{black}                               % color options 'blue' (default),
                                                  % 'orange', 'green', 'red', 'purple',
                                                  % 'grey' and 'black'

%% Another Options for colours
% \moderncvtheme[green]{classic}                % idem
% \definecolor{color0}{rgb}{0,0,0}% black
% \definecolor{color1}{rgb}{0.38,0.69,0.19}% green
% \definecolor{color2}{rgb}{0.45,0.45,0.45}% dark grey



\def\changemargin#1#2{\list{}{\rightmargin#2\leftmargin#1}\item[]}
\let\endchangemargin=\endlist
% [REF] http://tex.stackexchange.com/questions/588/how-can-i-change-the-margins-for-only-part-of-the-text

\usepackage{xstring}
\def\FormatName#1{%
  \IfSubStr{#1}{Xochicale}{\textbf{#1}}{#1}%
}
% Using my_unsrt.bst
% [REF] http://tex.stackexchange.com/questions/33330/make-one-authors-name-bold-every-time-it-shows-up-in-the-bibliography


\usepackage[scale=0.93]{geometry}% adjust the page margins
%\setlength{\hintscolumnwidth}{3cm} % if you want to change the width of the column with
%the dates

%\setlength{\makecvtitlenamewidth}{10cm} % for the 'classic' style, if you want to force
%the width allocated to your name and avoid line breaks. be careful though, the length is
%normally calculated to avoid any overlap with your personal info; use this at your own
%typographical risks...



\name{Miguel}{P. Xochicale}
\title{Curriculum Vitae -- December 2017}



% \quote{
%
% } % optional, remove / comment the line if not wanted


%----------------------------------------------------------------------------------
%            content
%
\begin{document}


%-----       resume       ---------------------------------------------------------
%
%
\makecvtitle



% %%%%%%%%%%%%%%%%%%%%%%%%%%%%%%%%%%%%%%%%%%%%%%%%%%%%%%%%%%%%%%%%%%%%%%%%%%%%%%%%%%%%%%%%%%%%%%%%%%%%%%%%%%%%%%%%
% \cvline{Scientific Focus}{\small \textbf{Human-Humanoid Imitation}:
%     human-humanoid interaction; nonlinear dynamics.}
% \cvline{Keywords}{\textbf{Human-Robot Interaction}, \textbf{Wearable Inertial Sensors}, \textbf{Human-Activity Recognition},  \textbf{Deep Learning}.}
%


\vspace{-15mm}


%%%%%%%%%%%%%%%%%%%%%%%%%%%%%%%%%%%%%%%%%%%%%%%%%%%%%%%%%%%%%%%%%%%%%%%%%
%%%%%%%%%%%%%%%%%%%%%%%%%%%%%%%%%%%%%%%%%%%%%%%%%%%%%%%%%%%%%%%%%%%%%%%%%
%%%
%%%
\section{Contact}
%
% For more icons
% http://mirror.ox.ac.uk/sites/ctan.org/fonts/fontawesome/doc/fontawesome.pdf

\begin{changemargin}{2.4cm}{0.5cm}
  \begin{minipage}{.4\textwidth}
    \begin{description}
     \item [\faPhone ]    +44 (0) 121 414 314 1 (UK)
     \item[\faEnvelope]  \href{mailto:perez.xochicale@gmail.com}{perez.xochicale@gmail.com}
     \item[\faHome]  \href{http://mxochicale.github.io}{http://mxochicale.github.io}
    % \item[Address:] South Parks Rd, OX1 3UD
     \end{description}
  \end{minipage}
  \begin{minipage}{.4\textwidth}
  \begin{description}
   \item[\faTwitter ] \href{https://twitter.com/_mxochicale}{@\textunderscore mxochicale}
   \item[\faGithub]  \href{https://github.com/mxochicale}{mxochicale}
  %  \item[\faLinkedin]
   \item[ORCID:] \href{http://orcid.org/0000-0002-8225-7517}{0000-0002-8225-7517}
   \end{description}
  \end{minipage}
\end{changemargin}
% \vspace{10pt}



%%%%%%%%%%%%%%%%%%%%%%%%%%%%%%%%%%%%%%%%%%%%%%%%%%%%%%%%%%%%%%%%%%%%%%%%%
%%%%%%%%%%%%%%%%%%%%%%%%%%%%%%%%%%%%%%%%%%%%%%%%%%%%%%%%%%%%%%%%%%%%%%%%%
%%%
%%%
\section{Research Interests}
\begin{changemargin}{2.4cm}{0.5cm}
I am interested in the fields of Human-Robot Interaction and Human Activity Recognition.
For my PhD, I am gaining deeper understanding of the variability
of human movements and facial expressions using Nonlinear Dynamics and Deep Learning.
% in order to create novel analysis and interpretation of signals collected
% through a network of Inertial Measurement Units.
\end{changemargin}



%%%%%%%%%%%%%%%%%%%%%%%%%%%%%%%%%%%%%%%%%%%%%%%%%%%%%%%%%%%%%%%%%%%%%%%%%
%%%%%%%%%%%%%%%%%%%%%%%%%%%%%%%%%%%%%%%%%%%%%%%%%%%%%%%%%%%%%%%%%%%%%%%%%
%%%
%%%
\section{Education}
\cventry{11/2014 -- Present}{Ph.D. in Human-Robot Interaction}{University of Birmingham}{UK}{}
  {Thesis: Automatic Classification of Movement Variability in the context of Human-Robot Interaction
  \href{https://github.com/mxochicale/PhD}{\faGithubAlt}
   \\ Advisors: Professor Chris Baber and  Professor Martin Russell  }

\cventry{09/2004 -- 09/2006}{M.Sc. in Electronics}
  {Instituto Nacional de Astrof\'isica, \'Optica y Electr\'onica}{M\'exico}{}
  {Thesis: Digital Filter FIR with less multipliers
  \href{https://github.com/mxochicale/publications/blob/master/Thesis/M.Sc./doc/MPXochicale_MScThesis-2016.pdf}{\faFilePdfO}
  \href{https://github.com/mxochicale/publications/tree/master/Thesis/M.Sc.}{\faGithubAlt}
  \\ Advisor: Gordana Jovanovic Dolecek}

\cventry{08/1999 -- 09/2004}{B.Eng. in Electronics}{Instituto Tecnol\'ogico de Puebla}{M\'exico}{}
  {Thesis: Speed control in LabVIEW for a two-degrees-of-freedom Robot.
  \href{https://github.com/mxochicale/publications/blob/master/Thesis/B.Eng./doc/MPXochicale_BachelorEngThesis-2003.pdf}{\faFilePdfO}
  \href{https://github.com/mxochicale/publications/tree/master/Thesis/B.Eng.}{\faGithubAlt}
  \\ Advisor: M.Sc. Jos\'e Esteban Torres Le\'on.}


%%%%%%%%%%%%%%%%%%%%%%%%%%%%%%%%%%%%%%%%%%%%%%%%%%%%%%%%%%%%%%%%%%%%%%%%%
%%%%%%%%%%%%%%%%%%%%%%%%%%%%%%%%%%%%%%%%%%%%%%%%%%%%%%%%%%%%%%%%%%%%%%%%%
%%%
%%%
\nocite{*}
\bibliographystyle{my_unsrt}
\bibliography{publications}



%%%%%%%%%%%%%%%%%%%%%%%%%%%%%%%%%%%%%%%%%%%%%%%%%%%%%%%%%%%%%%%%%%%%%%%%%
%%%%%%%%%%%%%%%%%%%%%%%%%%%%%%%%%%%%%%%%%%%%%%%%%%%%%%%%%%%%%%%%%%%%%%%%%
%%%
%%%
\section{Teaching Experience}

\cventry{08/2017--12/2017}{Teaching Associate}
{University of Birmingham}{UK}{}
{Engineering Maths 2. Lecturer: Professor Martin Russell}

\cventry{08/2017--12/2017}{Teaching Associate}
{University of Birmingham}{UK}{}
{Computing for Engineering. Lecturer: Dr Sridhar Pammu}

\cventry{01/2017--06/2017}{Teaching Associate}
{University of Birmingham}{UK}{}
{Matlab Laboratories. Lecturer: Dr Edward Tarte }

\cventry{08/2016--12/2016}{Teaching Associate}
{University of Birmingham}{UK}{}
{Computing for Engineering. Lecturer: Dr Sridhar Pammu}

\cventry{10/2014--12/2014}{Teaching Associate}
{University of Birmingham}{UK}{}
{Small Embedded Systems. Lecturer: Professor Chris Baber}

\cventry{08/2013--12/2013}{Teacher}
{Bilingual Hight School at TECMilenio University}{Puebla,M\'exico}{}
{Courses:
Information Technology \href{https://sites.google.com/site/perezxochicale/teaching/iit}{\faExternalLink},
Euclidian Geometry  \href{https://sites.google.com/site/perezxochicale/teaching/euclidean-geometry}{\faExternalLink}
and
Microsoft Office Access \href{https://sites.google.com/site/perezxochicale/teaching/moa}{\faExternalLink}
}

\cventry{Spring 2012 -- Autumn 2012}{Invited Lecturer in Mechatronic Engineering}
{Universidad Madero}{Puebla, M\'exico}{}
{Courses: Fundamentals of Automation
\href{https://sites.google.com/site/perezxochicale/digital-electronics}{\faExternalLink},
Industrial Electronics \href{https://sites.google.com/site/perezxochicale/ie}{\faExternalLink},
Research Projects \href{https://sites.google.com/site/perezxochicale/latex/thesistemplate}{\faExternalLink},
Metrology \href{https://sites.google.com/site/perezxochicale/metrology}{\faExternalLink},
Physics \href{http://goo.gl/fffnG}{\faExternalLink},
Computer Integrating Manufacturing, and Power Electronics
}

\cventry{Spring 2007 -- Spring 2012}{Invited Lecturer in Electronic Engineering}
{Universidad Iberoamericana Puebla}{M\'exico}{}
{Courses: Stochastic Processes Course
\href{https://sites.google.com/site/perezxochicale/stochastic-processes-course}{\faExternalLink},
Digital Signal Processing
\href{https://sites.google.com/site/perezxochicale/digital-signal-processing-course}{\faExternalLink}
and Analog Filters.
}

\cventry{08/2006 -- 06/2007}{Invited Lecturer in Mechatronic Engineering}
{Instituto Tecnol\'ogico Superior de Atlixco}{M\'exico}{}
{Courses: Electronics I, Numerical Methods, and Electricity and Magnetism. (January-June 2007.)
Electricity and Magnetism, and Electricity and Industrial Electronics ( August-December 2006)
}


%%%%%%%%%%%%%%%%%%%%%%%%%%%%%%%%%%%%%%%%%%%%%%%%%%%%%%%%%%%%%%%%%%%%%%%%%
%%%%%%%%%%%%%%%%%%%%%%%%%%%%%%%%%%%%%%%%%%%%%%%%%%%%%%%%%%%%%%%%%%%%%%%%%
%%%
%%%
\section{Professional Experience}

\cventry{02/2013 -- 08/2013}{Research Assistant}{INAOE's Robotics Laboratory}{M\'exico}{}
{Detailed achievements: I develop a Human-Robot Interaction Demo Dance which  was based on a ZSTAR3 Radio
Frequency single three-axis accelerometer and a Patrolbot mobile robot.
For the demo, I explored four gestures wearing the accelerometer in the left wrist
in order to create simple dance activities with the Patrolbot mobile robot.
For further information go to \href{https://sites.google.com/site/perezxochicale/projects/demodance}{\faExternalLink}.}

\cventry{01/2012 -- 01/2013}{Invited Lecturer}
{Universidad Madero}{Puebla, M\'exico}{}
{Detailed achievements:
I proposed and supervised the following students' projects: Haptic Referee Glove,
 Lightmetre and Pychometre using Arduino, Smart Irrigation, Persistent Of Vision Bicycle Wheel
 and a Delta Robot Structure.
\href{https://sites.google.com/site/perezxochicaleprojects/studentprojects}{\faExternalLink}
Additionally, I proposed and designed a Mechatronic Laboratory which includes a benchmark for
Mechatronic's laboratories in M\'exico and Puebla, a 3D layout design and minimal
requirements of hardware and software for the laboratory.
For further information go to \href{https://sites.google.com/site/perezxochicaleprojects/mechatronicslaboratorydesign}{\faExternalLink}.
}

\cventry{09/2003 -- 03/2004}{Research Internship}
{INAOE}{M\'exico}{}
{Detailed achievements:
I implemented a speed control for a two-degree-of-freedom Robot with Microcontrollers
PIC 16F84 and 16F877 that made communication via RS-232 using Virtual Instruments on LabVIEW.}

%%%%%%%%%%%%%%%%%%%%%%%%%%%%%%%%%%%%%%%%%%%%%%%%%%%%%%%%%%%%%%%%%%%%%%%%%
%%%%%%%%%%%%%%%%%%%%%%%%%%%%%%%%%%%%%%%%%%%%%%%%%%%%%%%%%%%%%%%%%%%%%%%%%
%%%
%%%
\section{Awards and Honours}

\cvitem{11/01/2017}{
I was selected to present advances of my Ph.D in the second forum of Mexican Talent Innovation Match MX
with my talk ``Towards the improvement of Healthy Ageing with Humanoid Robos''.
\href{https://github.com/mxochicale/InnovationMatchMX/tree/master/2017}{\faExternalLink}.
\href{https://github.com/mxochicale/InnovationMatchMX/blob/master/2017/presentation/IMMX-MA-0058.pdf}{\faFilePdfO}
}

\cvitem{16-18/06/2016}{
I won a shared first prize for presenting one of the two best posters at
the XIV Symposium of Mexican Students in the UK at University of Edinburgh.
\href{https://github.com/mxochicale/symposiummx/tree/master/2016}{\faExternalLink}.}

\cvitem{20-24/07/2015}{My project of a low-cost robot was selected amoung 125 applications received from 35 countries
and presented at the first international public entrepreneurship program in Mexico (MECATE 2015)
\href{http://let-emprendimientopublico.mx/en/portfolio_category/mecate-primera-generacion-en/}{\faExternalLink}
\href{https://www.youtube.com/watch?v=VjVGnwD422g}{\faYoutube}.}

\cvitem{11/2014-11/2018}{
Full Ph.D. Scholarship in the UK from the Mexican National Council on Science and Technology (CONACyT).}

\cvitem{25-27/05/2013}{
Markovito, a service robot, won the first place at the Mexican Tournament of Robotics 2013 in the cathegory at HOME
where I presented a Human-Robot Interation Dance Demo \href{https://www.youtube.com/watch?v=Kw-lZam_qZI}{\faYoutube}. }

\cvitem{09/2004-09/2006}{
Full M.Sc. Scholarship in M\'exico from the CONACyT. }

%%%%%%%%%%%%%%%%%%%%%%%%%%%%%%%%%%%%%%%%%%%%%%%%%%%%%%%%%%%%%%%%%%%%%%%%%
%%%%%%%%%%%%%%%%%%%%%%%%%%%%%%%%%%%%%%%%%%%%%%%%%%%%%%%%%%%%%%%%%%%%%%%%%
%%%
%%%
\section{Languages}
\cvitemwithcomment{English}{IETLS Band Score 6.0: Listening 6.0, Reading 7.0, Writing 6.0, Speaking 5.5.}{11/01/14}
\cvitemwithcomment{Spanish}{Native tongue}{}

%%%%%%%%%%%%%%%%%%%%%%%%%%%%%%%%%%%%%%%%%%%%%%%%%%%%%%%%%%%%%%%%%%%%%%%%%
%%%%%%%%%%%%%%%%%%%%%%%%%%%%%%%%%%%%%%%%%%%%%%%%%%%%%%%%%%%%%%%%%%%%%%%%%
%%%
%%%
\section{Technical Skills}
\cvitem{General}{
Deep Learning (e.g., TensorFlow);
Inertial Measurament Units(data collection and analysis);
Graphic design (Inkscape)
}
\cvitem{Programming}{R, python,
Robot Operating System (ROS),
C, C++,
Arduino, Processing, \LaTeX, the shell,
GNU-emacs,
GNU-Octave, MATLAB and LabVIEW.}

%%%%%%%%%%%%%%%%%%%%%%%%%%%%%%%%%%%%%%%%%%%%%%%%%%%%%%%%%%%%%%%%%%%%%%%%%
%%%%%%%%%%%%%%%%%%%%%%%%%%%%%%%%%%%%%%%%%%%%%%%%%%%%%%%%%%%%%%%%%%%%%%%%%
%%%
%%%
%%%%%%%%%%%%%%%%%%%%%%%%%%%%%%%%%%%%%%%%%%%%%%%%%%%%%%%%%%%%%%%%%%%%%%%%%%%%%%%%%%%%%%%%%%%%%%%%%%%%%%%%%%%%%%%%
\section{Scientific Engagement}



\cventry{2017--2018}{Webmaster and contributor of Machine Learning for Mexico}{}{ GitHub: \href{https://github.com/ML4MX}{\faGithubAlt} }{ Website: \href{https://ml4mx.github.io/website/}{\faExternalLink}}{}  % arguments 3 to 6 are optional

\cventry{2017--2018}{Coordinator of the Science Seminars for the Mexican Society at University of Birmingham, UK}
{}
{GitHub: \href{https://github.com/MexicanSocietyUoB}{\faGithubAlt}}
{Website: \href{https://mexicansocietyuob.github.io/seminars/}{\faExternalLink}}
{}
% arguments 3 to 6 are optional


%\cventry{2014-2018}
%{Outreach Activities: How to build low-cost robots for children}
%{}
%{at University of Birmingham, UK} 
%{GitHub: \href{https://github.com/librerobotics/printable-robots}{\faGithubAlt} }
%{}

\cventry{2013--2018}
{Founder of LibrE Robotics}
{a non-profit organization, to transfer knowledge of Educative Robotics for children to build conditions for a better world}
{GitHub: \href{https://github.com/librerobotics}{\faGithubAlt} }
{Website: \href{https://sites.google.com/site/LibreRobotics/}{\faExternalLink}}{}
{}  
% arguments 3 to 6 are optional

%\cventry{2017--2018}{Webmaster and contributor of Machine Learning for Mexico}{}
%


\end{document}
