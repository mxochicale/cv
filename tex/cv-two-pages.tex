\documentclass{mycv}

\name{Miguel Xochicale}
\researchinterest{
\\ My research interests are advancing AI tools for Medical Imaging, MedTech, SurgTech, Biomechanics and Clinical Translation.
}

\address{
Research Engineer, advancing AI-based Surgical Navigation tools\\
Advanced Research Computing Centre and WEISS \\
University College London, UK
}
\homepage{http://mxochicale.github.io}
\email{m.xochicale@ucl.ac.uk}
%\github{mxochicale}
\linkedin{mxochicale}

\begin{document}
\maketitle

%%%%%%%%%%%%%%%%%%%%%%%%%%%%%%%%%%%%%%%%%%%%%%
%\section{Research Interests}
%My research interests are in real-time and automatic signal and image processing techniques and AI-based fetal biomechanics.
%I have also experience in areas of Human-Robot Interaction, Movement Variability, Wearables in Medicine, Ultrasound-Guided Interventions, Medical Robotics, and Research Software Engineering.
%I have passion for real-time AI-based technologies for Healthcare.

%\small
\footnotesize

\section{Education}
\subsection{Ph.D.\ in Computer Engineering at The University of Birmingham}[Birmingham, UK]
% \vspace{-\parskip}%
\begin{itemize}[label={}]
  %\item The University of Birmingham \printdate{Nov 2014~--~Nov 2018}
  \item Thesis: Nonlinear Analysis to Quantify Movement Variability in Human-Humanoid Interaction. \printdate{Nov 2014~--~Nov 2018}
  \item Supervisors: Professor Chris Baber and  Professor Martin Russell
%   \item	Thesis submission: 26/10/2018. Passed Viva: 11/01/2019. Awarded PhD degree: 12/07/2019. \\
  \item	Awarded PhD degree: 12/07/2019. Thesis: \href{https://doi.org/10.5281/zenodo.3384145}{\faFilePdf}
	GitHub: \href{http://doi.org/10.5281/zenodo.3384281}{\faGithub*}
	Website: \href{https://mxochicale-phd.github.io/site/}{\faExternalLink*} 
\end{itemize}

\subsection{M.Sc. in Signal Processing at Institute of Astrophysics, Optics and Electronics (INAOE)}[Puebla, M\'exico]
% \vspace{-\parskip}%
\begin{itemize}[label={}]
%   \item Institute of Astrophysics, Optics and Electronics (INAOE) \printdate{Sep 2004~--~Sep 2006}
  \item Thesis: Design of digital filters with fewer multipliers. \printdate{Sep 2004~--~Sep 2006}
  \item Supervisor: Dr. Gordana Jovanovic Dolecek 
  \item Awarded MSc degree: 01/11/2006.
  	Thesis: \href{https://github.com/mxochicale/publications/blob/master/thesis/M.Sc./doc/MPXochicale_MScThesis-2016.pdf}{\faFilePdf}
	GitHub: \href{https://github.com/mxochicale/publications/tree/master/thesis/M.Sc.}{\faGithub*}
\end{itemize}

% \subsection{Puebla Institute of Technology}[Puebla, M\'exico]
% \vspace{-\parskip}%
% \begin{itemize}[label={}]
%   \item B.Eng. in Electronics \printdate{Aug 1999~--~Sep 2004}
%   \item Thesis: Speed control for a two-degrees-of-freedom Robot in LabVIEW.
%   \item Supervisor: M.Sc. Jos\'e Esteban Torres Le\'on.
%   \item	Links: Thesis: \href{https://github.com/mxochicale/publications/blob/master/thesis/B.Eng./doc/MPXochicale_BachelorEngThesis-2003.pdf}{\faFilePdfO} Github: \href{https://github.com/mxochicale/publications/tree/master/thesis/B.Eng.}{\faGithub*}
% \end{itemize}

%%%%%%%%%%%%%%%%%%%%%%%%%%%%%%%%%%%%%%%%%%%%%%
\section{Research Experience}
\subsection{Research Engineer at University College London, UK}[Oct 2022~--~Present]
\begin{itemize}
  \item Collaborators: \href{https://scholar.google.co.uk/citations?user=-rD4cJIAAAAJ}{Stephen Thompson},  \href{https://iris.ucl.ac.uk/iris/browse/profile?upi=TMDOW59}{Dr Thomas Dowrick} and \href{https://iris.ucl.ac.uk/iris/browse/profile?upi=MJCLA42}{Prof Matt Clarkson}
  \item 
I am advancing AI-based Surgical Navigation tools with Python and CUDA programming languages via GitHub.
%Additionally, I am leading the preparation of one proceeding and one manuscript in the area of medical imaging with real-time deep learning techniques.
%All previous activities in collaboration with renowned clinicians and engineers in KCL, University of Oxford and University of Melbourne. 
See more at \href{https://github.com/SciKit-Surgery}{(\faExternalLink*)}.
\end{itemize}


\subsection{Research Associate in Real-time AI-based Ultrasound Imaging at King's College London, UK}[Sep 2021~--~Sep 2022]
%\begin{positions}
%  \entry{Research Associate in Real-time AI-based Ultrasound Imaging}{Sep 2021~--~Present}
%\end{positions}
\begin{itemize}
  \item PIs: \href{http://kclmmag.org/}{Dr Andrew King} and  \href{https://gomezalberto.github.io/}{Dr. Alberto Gomez}
  \item 
I scientifically contributed to automatic biometric recognition of electrocardiography ultrasound data using real-time deep learning techniques with Python, CUDA, C++ and Qt programming languages via GitHub.
%Additionally, I am leading the preparation of one proceeding and one manuscript in the area of medical imaging with real-time deep learning techniques.
%All previous activities in collaboration with renowned clinicians and engineers in KCL, University of Oxford and University of Melbourne. 
See more at \href{http://vital.oucru.org/major-partners/kings-college-london/}{(\faExternalLink*)}.
\end{itemize}

\subsection{Research Associate in Software and Hardware Engineering at King's College London, UK}[Apr 2019~--~Aug 2021]
%\begin{positions}
%  \entry{Research Associate in Software and Hardware Engineering}{Apr 2019~--~Aug 2021}
%\end{positions}
\begin{itemize}
  \item PIs: \href{https://www.purlkcl.org/}{Dr. Wenfeng Xia} and \href{https://cai4cai.ml}{Prof.~Tom Vercauteren}
  \item 
I pushed forward the state-of-the-art of Ultrasound-Guidance Interventions, 
	contributing to the development of a needle tip tracking system, real-time ultrasound image processing, 
	quality management system for clinical translation of medical devices, and public engagement activities.
%	Similarly, I developed validation experiments with linear stages under Windows and GNU/Linux OSs,
%	designed electronic PCBs and design CAD pieces for 3D printing holders, characterised ultrasonic transducers,
%    operate clinical ultrasound devices and contributing to a Python library via GitHub following QMS.
%	All the previous activities in collaboration with an amazing 
%	team of renowned clinicians, engineers, QMS specialists and researchers in KCL and UCL.
%	Additionally, I lead the preparation of one manuscript in a high-impact factor journal.
	See more at \href{https://cai4cai.ml/author/miguel-xochicale/}{(\faExternalLink*)}.
\end{itemize}

% \newpage

% \subsection{Research Assistant in Robotics at INAOE's Robotics Laboratory, Puebla, M\'exico}[Feb 2013~--~Aug 2013]
% %\begin{positions}
% %  \entry{Research Assistant in Robotics}{Feb 2013~--~Aug 2013}
% %\end{positions}
% \begin{itemize}
% \item Advisor: Dr. Ang\'elica Mu\~noz Mel\'endez
% \item I developed a Human-Robot Interaction application for dancing activities based on a Patrolbot mobile robot and a single three-axial accelerometer. 
% %For the demo, I explored four hand gestures where the user worn the accelerometer at his/her 
% %left wrist in order to create simple dance activities with the mobile robot. 
% \href{https://sites.google.com/site/perezxochicale/projects/demodance}{(See documents and code: \faExternalLink)}.
% \end{itemize}

%\subsection{Madero University}[Puebla, M\'exico]
%\begin{positions}
%  \entry{Teaching Lecturer in Mechatronic Engineering}{Jan 2012~--~Jan 2013}
%\end{positions}
%\begin{itemize}
%	\item I proposed and supervised the following students projects: Haptic Referee Glove,
%	Lightmetre and Pychometre Sensors, Smart Irrigation, Persistent Of Vision Bicycle Wheel
%	and a Delta Robot Structure 
%	\href{https://sites.google.com/site/perezxochicaleprojects/studentprojects}{(See documents and code: \faExternalLink)}.
%	\item I proposed and designed a Mechatronic Laboratory which includes: 
%	(i) a benchmark for laboratories in mechatronics in M\'exico and Puebla, 
%	(ii) a 3D layout design and 
%	(iii) minimal requirements of hardware and software for the laboratory
%	\href{https://sites.google.com/site/perezxochicaleprojects/mechatronicslaboratorydesign}{(See documents and layout: \faExternalLink)}.
%\end{itemize}
%
%\subsection{INAOE}[Puebla, M\'exico]
%\begin{positions}
%  \entry{Research Internship in Robotics}{Sep 2003~--~Mar 2004}
%\end{positions}
%\begin{itemize}
%  \item  I implemented a speed control for a two-degree-of-freedom robot with microcontrollers
%PIC 16F84 \& 16F877 which communicated via RS-232 to LabVIEW's Virtual Instruments.
%\end{itemize}
%


%%%%%%%%%%%%%%%%%%%%%%%%%%%%%%%%%%%%%%%%%%%%%%
\section{Publications}
\publications{publications.bib}
%%%%%%%%%%%%%%%%%%%%%%%%%%%%%%%%%%%%%%%%%%%%%%
\section{Talks and Posters}
\publications{talks-and-posters.bib}
%%%%%%%%%%%%%%%%%%%%%%%%%%%%%%%%%%%%%%%%%%%%%%



%%%%%%%%%%%%%%%%%%%%%%%%%%%%%%%%%%%%%%%%%%%%%%
\section{Supervision and Teaching Experience}

\subsection{University College London}[Jan 2023~--~Present]
\begin{itemize}
	\item Teaching Associate: AI in Healthcare Group projects. Lecturer: Prof. Paul Taylor \printdate{Jan 2023~--~Present}
	\\ I led two group projects on \textit{Fetal Brain Ultrasound Imaging synthesis with Diffusion Models}.
	\item Supervision: Xiaoning Zhu, MPhil in Artificial Intelligence Enabled Healthcare
		\textit{Project: Automatic Medical Image Reporting.}
		\printdate{Jan 2023~--~Aug 2023}
	\item Supervision: Qingyu Yang, MPhil in Artificial Intelligence Enabled Healthcare
		\textit{Project: High-resolution Fetal Brain Ultrasound Imaging synthesis with Diffusion Models.}
		\printdate{May 2023~--~Aug 2023}
\end{itemize}

\subsection{King's College London}[Jan 2020~--~September 2022]
\begin{itemize}
	\item 	Supervision: Pablo Prieto Roca and Samuel Eyob. 
		\textit{KURF projects on 3DGANs for fetal US imaging}. 
		\printdate{Jun 2022~--~Aug 2022} 
	\item 	Supervision: Tsz Yan (Goosie) Leung, MSc in Medical Engineering and Physics. 
		\textit{Project: Simple US guidance intervention.} 
		\printdate{Feb 2021~--~Aug 2022} 
	\item 	Supervision: Thea Bautista, M. Eng. in Biomedical Engineering. 
		\textit{Project: DCGANs for fetal US imaging.} 
		\printdate{Oct 2021~--~May 2022} 
	\item 	Supervision: Guilherme Gomes de Figueiredo and Amal Hussein. 
		\textit{KURF projects on DCGANs for US imaging.}
		\printdate{Jun 2021~--~Aug 2021} 
	\item 	Supervision: Alexander Mitton, M.Sc. in Medical Engineering and Physics. 
		\textit{Project: Vibro-tactile stimulator for dystonia.}
		\printdate{Jan 2020~--~Sep 2020} 
	\item Teaching Associate: Medical Robotics. Lecturer: Dr. Alejandro Granados \printdate{Jan 2022~--~Apr 2022}
	\item Teaching Associate: Medical Robotics. Lecturer: Dr. Hongbin Liu \printdate{Jan 2021~--~Apr 2021}
	\item Teaching Associate: Medical Robotics. Lecturer: Dr. Christos Bergeles \printdate{Jan 2020~--~Apr 2020}
\end{itemize}

% \subsection{King's College London}[London, UK]
% \begin{positions}
%   \entry{Supervision}{Jan 2020~--~Present}
% \end{positions}



%Tsz Yan (Goosie) Leung, MSc in Medical Engineering and Physics \printdate{Feb 2021~--~Aug 2022} 
%Title: "Towards simple and effective ultrasound-guidance procedures" \\
%Co-supervisors: A. King and A. Gomez

%Thea Bautista, M. Eng. in Biomedical Engingeering \printdate{Oct 2021~--~May 2022} 
%Title: GAN-based synthetic ultrasound imaging for fetal development \\
%Co-supervisors: H. Kerdegari, L. Peralta-Pereira, and R. Aughwane

%Guilherme Gomes de Figueiredo and Amal Hussein, Summer projects \printdate{Jun 2021~--~Aug 2021} 
%Summer Project: synthetic ultrasound imagining with AI

%Student: Amal Hussein \printdate{Jun 2021~--~Aug 2021} \\
%Summer Project: Ultrasound-guidance simulator

%Alexander Mitton, M.Sc. in Medical Engineering and Physics \printdate{Jan 2020~--~Sep 2020} 
%M.Sc. Project: Vibro-tactile stimulator for dystonia research \\
%Co-supervisors: C. Bergeles, V. Mcclelland and A. Worley



\subsection{The University of Birmingham}[Jun 2018~--~ Dec 2018]
% \subsection{King's College London}[Jan 2020~--~Present]

% % \newpage
% \begin{positions}
% 	\entry{Supervision}{Jun 2018~--~ Dec 2018}
% \end{positions}
% \begin{itemize}
% \item Dinghuang Zhang, M.Sc. Project: Tools for Human-Humanoid Collaboration
% %Co-supervisor: Chris Baber. 
% \end{itemize}

% \begin{positions}
%   \entry{Teaching Associate}{Aug 2014~--~Apr 2018}
% \end{positions}
\begin{itemize}
\item 	Supervision: Dinghuang Zhang, M.Sc. in Computer Engineering. 
	\textit{Project: Tools for human-humanoid collaboration}
	\printdate{Aug 2018 -- Dec 2018} %Project: Tools for Human-Humanoid Collaboration 
\item Teaching associate: Engineering Maths 2. Lecturers: Prof. Martin Russell, Dr Carl Anthony \printdate{Aug 2017 -- Dec 2017; Jan 2018 -- Apr2018}
%\item Engineering Maths 2. Lecturer: Prof. Martin Russell \printdate{Aug 2017 -- Dec 2017}
\item Teaching Associate: Matlab Laboratories. Lecturer: Dr Edward Tarte  \printdate{Jan 2017 -- Apr 2017}
%\item Computing for Engineering. Lecturer: Dr Sridhar Pammu  \printdate{Aug 2017 -- Dec 2017}
\item Teaching Associate Computing for Engineering. Lecturer: Dr Sridhar Pammu  \printdate{Aug 2016 -- Dec 2016; Aug 2017 -- Dec 2017}
\item Teaching Associate: Small Embedded Systems. Lecturer: Prof. Chris Baber  \printdate{Aug 2016 -- Dec 2016} 
\end{itemize}

%\subsection{Teaching Associate at Mexican Institutions}[Aug 2006~--~Dec 2013]
%%\begin{positions}
%%  \entry{Teaching Associate}{Aug 2013~--~Dec 2013}
%%\end{positions}
%\begin{itemize}
%\item Teaching Associate at Bilingual Hight School TECMilenio University, Puebla, M\'exico. \printdate{Aug 2013~--~Dec 2013} 
%\item Teaching Associate in Mechatronic Eng at Universidad Madero,Puebla, M\'exico. \printdate{Jan 2012~--~Dec 2012}
%\item Teaching Associate in Electronic Eng. at Universidad Iberoamericana Puebla,Puebla, M\'exico. \printdate{Jan 2007~--~Dec 2011}
%\item Teaching Associate in Mechatronic Eng. at Instituto Tecnol\'ogico Superior de Atlixco,Puebla, M\'exico. \printdate{Aug 2006~--~Jun 2007}
%\end{itemize}
%

%\subsection{Bilingual Hight School TECMilenio University}[Puebla, M\'exico]
%\begin{positions}
%  \entry{Teaching Associate}{Aug 2013~--~Dec 2013}
%\end{positions}
%\begin{itemize}
%	\item
%	Information Technology \href{https://sites.google.com/site/perezxochicale/teaching/iit}{\faExternalLink}, 
%	Euclidean Geometry  \href{https://sites.google.com/site/perezxochicale/teaching/euclidean-geometry}{\faExternalLink}, and 
%	Microsoft Office Access \href{https://sites.google.com/site/perezxochicale/teaching/moa}{\faExternalLink}
%\end{itemize}
%


%\subsection{Universidad Madero}[Puebla, M\'exico]
%\begin{positions}
%  \entry{Teaching Associate in Mechatronic Eng.}{Jan 2012~--~Dec 2012}
%\end{positions}
%\begin{itemize}
%	\item Fundamentals of Automation \href{https://sites.google.com/site/perezxochicale/digital-electronics}{\faExternalLink}, 
%	Industrial Electronics \href{https://sites.google.com/site/perezxochicale/ie}{\faExternalLink},
%	Research Projects \href{https://sites.google.com/site/perezxochicale/latex/thesistemplate}{\faExternalLink},
%	Metrology \href{https://sites.google.com/site/perezxochicale/metrology}{\faExternalLink},
%	Physics \href{http://goo.gl/fffnG}{\faExternalLink}, and 
%	Computer Integrating Manufacturing, and Power Electronics
%\end{itemize}
%
%\subsection{Universidad Iberoamericana Puebla}[Puebla, M\'exico]
%\begin{positions}
%  \entry{Teaching Associate in Electronic Eng.}{Jan 2007~--~Dec 2011}
%\end{positions}
%\begin{itemize}
%	\item Stochastic Processes \href{https://sites.google.com/site/perezxochicale/stochastic-processes-course}{\faExternalLink}, 
%	Digital Signal Processing \href{https://sites.google.com/site/perezxochicale/digital-signal-processing-course}{\faExternalLink}, 
%	and Analog Filters.
%\end{itemize}
%
%\subsection{Instituto Tecnol\'ogico Superior de Atlixco}[Puebla, M\'exico]
%\begin{positions}
%  \entry{Teaching Associate in Mechatronic Eng.}{Aug 2006~--~Jun 2007}
%\end{positions}
%\begin{itemize}
%	\item (01/2007 - 06/2007) Electronics I, Numerical Methods, and Electricity and Magnetism. 
%	\item (08/2006 - 12/2006) Electricity and Magnetism, and Electricity and Industrial Electronics 
%\end{itemize}
%



%%%%%%%%%%%%%%%%%%%%%%%%%%%%%%%%%%%%%%%%%%%%%%
\section{Grants, Awards and Honours}
\begin{itemize}

\item My M.Sc. student, Thea Bautista received the Maurice Wilkins Prize at KCL for the best MEng Individual Research Project in 2022 \href{https://twitter.com/_mxochicale/status/1564571097932402688}{\faExternalLink*} \printdate{(30/08/2022)}.

\item King's Public Engagement grant for the project "FETUS: Finding a fETus with an Ultrasound Simulator" led by myself and in collaboration with Fang-Yu Lin and Shu Wang \href{https://cai4cai.ml/post/2021-01-07-miguelpegrant/}{\faExternalLink*} \printdate{(07/01/2021 - 07/01/2022)}

\item My M.Sc. student, Alexander Mitton, won the Outstanding Individual Project award 
%for his M.Sc. project which I was the main supervisor
%, on designing a wearable, vibrotactile stimulation device for patients with dystonia
 \href{https://www.kcl.ac.uk/news/mscmres-healthcare-technologies-award-student-prizes-for-outstanding-performance-and-contributions-to-student-life}{\faExternalLink*} \printdate{(15/10/2020)}

\item King's Health Partners grant for the project "Sensory system abnormalities in childhood dystonia" lead by Verity McClelland and in collaboration with Carlos Seneci \href{https://kclpure.kcl.ac.uk/portal/en/persons/miguel-angel-perez-xochicale(cca72683-31b7-496a-8aeb-181fd9d6a8f3)/projects.html}{\faExternalLink*} \printdate{(14/04/2020 - 9/06/2020)}

\item ``Towards Healthy Ageing with Humanoid Robots'' was selected for a talk at the 2nd forum of Mexican Talent 2017 \href{https://github.com/mxochicale/InnovationMatchMX/tree/master/2017}{\faExternalLink*} \href{https://github.com/mxochicale/InnovationMatchMX/blob/master/2017/presentation/IMMX-MA-0058.pdf}{\faFilePdf} \href{https://www.youtube.com/watch?v=wNWzpdXdm5U}{\faYoutube} \printdate{11/01/2017}

\item I won the best poster award at the XIV Symposium of Mexican Students in the UK at University of Edinburgh \href{https://github.com/mxochicale/symposiummx/tree/master/2016}{\faExternalLink*} \printdate{16-18/06/2016} 

\item My project of a low-cost robot was selected among 125 applications received from 35 countries and presented at the 1st international public entrepreneurship program in Mexico (MECATE 2015). \href{http://let-emprendimientopublico.mx/en/portfolio_category/mecate-primera-generacion-en/}{\faExternalLink*} \href{https://www.youtube.com/watch?v=VjVGnwD422g}{\faYoutube} \printdate{20-24/07/2015}

\item I won a Ph.D. scholarship by the Mexican National Council on Science and Technology and University of Birmingham. \printdate{11/2014-11/2018}

\item Markovito's team won the first place at the Mexican Tournament of Robotics 2013 in the category at HOME, presenting a Human-Robot Interaction Dance Demo. \href{https://www.youtube.com/watch?v=Kw-lZam_qZI}{\faYoutube} \printdate{25-27/05/2013}

%\item I won a M.Sc. scholarship by the Mexican National Council on Science and Technology. \printdate{09/2004-09/2006}
\end{itemize}

%%%%%%%%%%%%%%%%%%%%%%%%%%%%%%%%%%%%%%%%%%%%%%
\section{Skills}
\begin{description}

\item[Programming and software] Python[2014-present],
GNU/Linux Operating System user (e.g. OpenSuse, Debian and Ubuntu)[2005-present], 
R[2013-present], 
Robot Operating System (ROS)[2016-present],
GNU-Octave (as well as MatLab)[2009-present],
\LaTeX [2006-present], 
C and C++[2015-present],
Processing[2012-present], 
the shell[2010-present], 
GNU-emacs[2010-present],
vim[2016-present], 
pandoc[2017-present],
open-source enthusiast at GitHub (\href{https://github.com/mxochicale}{@mxochicale})[2015-present], 
continuous integration and continuous delivery [2019-present], 
CUDA Programming [2021-present], and
python-based linting [2022-present].

\item[Tools and hardware] 
Single-board computers and microcontrollers (e.g. NVIDIA Jetson Nano, RaspberryPi, BeagleBone, Arduino and PIC)[2010-present],
Inertial Measurement Units (e.g. calibration, collection and data analysis)[2013-present], 
Web design (e.g. Github pages, Jekyll)[2015-present], 
Graphic design (e.g. Inkscape, GIMP)[2014-present],
CAD design (e.g. Autodesk inventor, blender, FreeCAD, and onshape)[2015-present], 
Artificial Neural Networks (e.g. PyTorch, and TensorFlow)[2017-present], 
3D printing (e.g., flsun, cura) [2019-present], 
Video framegrabbers (e.g. PCI and usb from ephipan), and 
Clinical Ultrasound Devices (e.g. Voluson E10, Philips EPIC, convex and linear probes), and 
Medical imaging (e.g. 3D slicer, ITK-SNAP).

\item[QMS] 
Control documents with standard compliances following:
Good Machine Learning Practice by FDA, Health Canada, and MHRA;
IEC 62304 Medical Device Software Standard;
BS EN 82304-1:2017 Health Software;
IEC 60601-1 Medical electrical equipment; 
BS EN 61391-2:2010 Ultrasonics;, and
BSI 60825-14 Safety of laser products.
%IEC 60601-1-4 and BS EN 60601-2-5; 

\item[Languages] Spanish[Native], English[Fluent] and interested in learning Chinese.
 
\end{description}

%%%%%%%%%%%%%%%%%%%%%%%%%%%%%%%%%%%%%%%%%%%%%%
\section{Outreach activities and scientific engagement}
%\subsection{Outreach activities and scientific engagement}[]
%\subsection{King's College London}[London, UK]
%\begin{positions}
%  \entry{Outreach activities and scientific engagement}{Sep 2019~--~Present}
%\end{positions}
\begin{itemize}
\item Lead organiser of "Open-Source Software for Surgical Technologies" at the Hamlym Symposium on Medical Robotics 2023
\href{https://www.hamlynsymposium.org/events/open-source-software-for-surgical-technologies/}{\faExternalLink*}. 
\printdate{January 2023 -- July 2023}.


\item Co-organised of journal club for computer vision and deep learning at Advanced Research Computing Centre \printdate{01-01-2023 -- present}.
\item Co-organised of activities for SciKit-Surgery tools for the CMICHACKS hackathon: \href{https://cmic-ucl.github.io/CMICHACKS//}{\faExternalLink*}. \printdate{10/11-11-2022}.
\item Co-organised of events at the Early Career Researcher Network at the BMEIS (monthly meetings \& writing workshops). \printdate{01-01-2021 - 01-09-2022}.
\item Participation in the Westminster Enterprise Week to engage students aged 14-18 to Biomedical Engineering. \printdate{10-11-2021}
\item Participant in the STEAM WEEK organised by the City Westminster Council to engage students aged 14-18 to STEAM \href{https://twitter.com/_mxochicale/status/1374407825607200769}{\faTwitter} \printdate{23-03-2021}
\item Alexandra Lautarescu and I organised the Reproducible, Interpretable, Open, \& Transparent Science Club at BMEIS \printdate{02-2020 -- 06-2020}
\item For the event In2ScienceUK, I shared my scientific journey to young scientist on how they can become better scientist.  \printdate{20-08-2019}
\item For the New Scientist Live, I showcased software that helps doctors to create 3D models of brain tumors using AI. \printdate{09-2019}
%\end{itemize}

%\subsection{Outreach activities and scientific engagement at University of Birmingham, UK}[Aug 2014~--~Jun 2018]
%\subsection{University of Birmingham}[Birmingham, UK]
%\begin{positions}
%  \entry{Outreach activities and scientific engagement}{Aug 2014~--~Jun 2018}
%\end{positions}

%\begin{itemize}
\item  Finalist at the Three Minute Thesis Competition 2018. Video: \href{https://www.youtube.com/watch?v=07ewRYcS-0g}{\faYoutube} and 
GitHub: \href{https://github.com/mxochicale/3mt}{\faGithub*} \printdate{05-2018}

\item Research Poster Conference for 
(2015) \href{https://github.com/mxochicale/PhD/blob/master/posters/Research_Poster_Conference_UoB/2015/poster/poster.pdf}{\faImage}, 
(2016) \href{https://github.com/mxochicale/PhD/blob/master/posters/Research_Poster_Conference_UoB/2016/poster/poster.pdf}{\faImage}, and  
(2018) \href{https://github.com/mxochicale/PhD/blob/master/posters/Research_Poster_Conference_UoB/2018/poster/main/map479-poster-uob2018.pdf}{\faImage}.
GitHub: \href{https://github.com/mxochicale/PhD/tree/master/posters/Research_Poster_Conference_UoB}{\faGithub*}.
\printdate{2016 to 2018}
\item Demoing Human-Robot Activities at the Undergraduate Open Days. GitHub: \href{https://github.com/mxochicale/opendayuob-hridemo}{\faGithub*}. \printdate{2014--2018}
\item Coordinator of the Science Seminars for the Mexican Society.  GitHub: \href{https://github.com/MexicanSocietyUoB}{\faGithub*}, Website: \href{https://mexicansocietyuob.github.io/seminars/}{\faExternalLink*}. \printdate{2017--2018}
%\end{itemize}


%\subsection{Other outreach activities and scientific engagement}[2013~--~Present]

%\subsection{AIR4Children}[M\'exico \& UK]
%\begin{itemize}

\item Building Artificial Intelligence and Robotics for Children. 
Twitter: \href{https://twitter.com/air4children}{\faTwitter @air4children} 
GitHub: \href{https://github.com/air4children}{\faGithub* @air4chidlren} 
\printdate{2019--Present} 

%\item Creation Libre Robotics, a non-profit organization aiming to freely transfer knowledge in Robotics to Mexican children.  Website: \href{https://sites.google.com/site/LibreRobotics/}{\faExternalLink}  \printdate{2013 -- 2017}

%\item Developer of the Website "Machine Learning for M\'exico"
%GitHub: \href{https://github.com/ML4MX}{\faGithub*}, Website: \href{https://ml4mx.github.io/website/}{\faExternalLink} 
%\printdate{2013 -- 2018}

\end{itemize}


% %%%%%%%%%%%%%%%%%%%%%%%%%%%%%%%%%%%%%%%%%%%%%%
% \section{References}

%% NOTE. Make the following changes tex/mycv.cls
%%top=10mm, %% Top margin makes a bit of space in the page!
%% bottom=10mm, %% Bottom margin makes a bit of space in the page!
%%% Commenting the following three lines create a bit of more space in the page
%%\fancyfoot[L]{\textcolor{Gray}{\@date}}
%%\fancyfoot[C]{\textcolor{Gray}{\@name~~~$\cdot$~~~Curriculum Vitae}}
%%\fancyfoot[R]{\textcolor{Gray}{\thepage}}

% \begin{itemize}
% 	\item \textbf{Name Surname}, Professor of , Instutitution, email@.com
% 	\item \textbf{Name Surname}, Professor of , Instutitution, email@.com
% 	\item \textbf{Name Surname}, Lecturer of , Instutitution, email@.com
% \end{itemize}


\end{document}
