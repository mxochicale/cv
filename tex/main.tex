\documentclass{mycv}

\name{Miguel Xochicale}
\address{
School of Biomedical Engineering and Imaging Sciences \\ 
Department of Biomedical Engineering \\
King's College London, UK
}
\homepage{http://mxochicale.github.io/}
\email{miguel.xochicale@kcl.ac.uk}
%\github{mxochicale}
\linkedin{mxochicale}

\begin{document}
\maketitle

\section{Research \\ Interests}
My research interests are in real-time and automatic Signal and Image Processing techniques in the context of Human-Robot Interaction, Movement Variability, Wearables in Medicine, Ultrasound-Guided Interventions, Medical Robotics, Fetal Biomechanics, and Research Software Engineering.
I have passion for real-time AI and biomechanics technologies for Healthcare.

\section{Education}
\subsection{The University of Birmingham}[Birmingham, UK]
\vspace{-\parskip}%
\begin{itemize}[label={}]
  \item Ph.D.\ in Computer Engineering \printdate{Nov 2014~--~Nov 2018}
  \item Thesis: Nonlinear Analysis to Quantify Movement Variability in Human-Humanoid Interaction.   
  \item Supervisors: Professor Chris Baber and  Professor Martin Russell
  \item	Thesis submission: 26/10/2018. Passed Viva: 11/01/2019. Awarded PhD degree: 12/07/2019. \\
	Links: Thesis: \href{https://doi.org/10.5281/zenodo.3384145}{\faFilePdfO}
	Github: \href{http://doi.org/10.5281/zenodo.3384281}{\faGithubAlt}
	Website: \href{https://mxochicale-phd.github.io/site/}{\faExternalLink} 
\end{itemize}

\subsection{Institute of Astrophysics, Optics and Electronics (INAOE)}[Puebla, M\'exico]
\vspace{-\parskip}%
\begin{itemize}[label={}]
  \item M.Sc. in Signal Processing \printdate{Sep 2004~--~Sep 2006}
  \item Thesis: Design of digital filters with fewer multipliers
  \item Supervisor: Dr. Gordana Jovanovic Dolecek
  \item	Links: 
	Thesis: \href{https://github.com/mxochicale/publications/blob/master/thesis/M.Sc./doc/MPXochicale_MScThesis-2016.pdf}{\faFilePdfO}
	Github: \href{https://github.com/mxochicale/publications/tree/master/thesis/M.Sc.}{\faGithubAlt}
\end{itemize}

\subsection{Puebla Institute of Technology}[Puebla, M\'exico]
\vspace{-\parskip}%
\begin{itemize}[label={}]
  \item B.Eng. in Electronics \printdate{Aug 1999~--~Sep 2004}
  \item Thesis: Speed control for a two-degrees-of-freedom Robot in LabVIEW.
  \item Supervisor: M.Sc. Jos\'e Esteban Torres Le\'on.
  \item	Links: Thesis: \href{https://github.com/mxochicale/publications/blob/master/thesis/B.Eng./doc/MPXochicale_BachelorEngThesis-2003.pdf}{\faFilePdfO} Github: \href{https://github.com/mxochicale/publications/tree/master/thesis/B.Eng.}{\faGithubAlt}
\end{itemize}

\section{Professional \\ Experience}

\subsection{King's College London}[London UK]
\begin{positions}
  \entry{Research Associate in Real-time AI Ultrasound Imaging}{Sep 2021~--~Present}
\end{positions}
\begin{itemize}
  \item PI: \href{https://gomezalberto.github.io/}{Dr. Alberto Gomez}
  \item
I am scientifically contributing to automatic biometric recognition of Electrocardiography ultrasound data using real-time deep learning techniques and frameworks with Python-based and Qt programming languages via GitHub.
Additionally, I am leading the preparation of one proceeding and one manuscript in the area of medical imaging with real-time deep learning techniques.
All previous activities in collaboration with renowned clinicians and engineers in KCL, University of Oxford and University of Melbourne. 
See more at \href{http://vital.oucru.org/major-partners/kings-college-london/}{(\faExternalLink)}.
\end{itemize}

\subsection{King's College London}[London UK]
\begin{positions}
  \entry{Research Associate in Software and Hardware Engineering}{Apr 2019~--~Aug 2021}
\end{positions}
\begin{itemize}
  \item PI: \href{https://cai4cai.ml}{Prof.~Tom Vercauteren}
  \item I pushed forward the state-of-the-art of Ultrasound-Guidance Interventions
	where was involved in the development of a needle tip tracking system, real-time ultrasound image processing, 
	quality management system (QMS) for clinical translation of medical devices, and public engagement activities.
	Similarly, I developed validation experiments with linear stages under Windows and GNU/Linux OSs,
	designed electronic PCBs and design CAD pieces for 3D printing holders, characterised ultrasonic transducers,
    operate clinical ultrasound devices and contributing to a Python library via GitHub following QMS.
	All the previous activities in collaboration with an amazing 
	team of renowned clinicians, engineers, QMS specialists and researchers in KCL and UCL.
	Additionally, I lead the preparation of one manuscript in a high-impact factor journal.
	See more at \href{https://cai4cai.ml/author/miguel-xochicale/}{(\faExternalLink)}.
\end{itemize}

\newpage

\subsection{INAOE's Robotics Laboratory}[Puebla, M\'exico]
\begin{positions}
  \entry{Research Assistant in Robotics}{Feb 2013~--~Aug 2013}
\end{positions}
\begin{itemize}
  \item Advisor: Dr. Ang\'elica Mu\~noz Mel\'endez
  \item I developed a Human-Robot Interaction application for dancing activities based on 
a Patrolbot mobile robot and a single three-axial accelerometer. 
%For the demo, I explored four hand gestures where the user worn the accelerometer at his/her 
%left wrist in order to create simple dance activities with the mobile robot. 
\href{https://sites.google.com/site/perezxochicale/projects/demodance}{(See documents and code: \faExternalLink)}.
\end{itemize}

\subsection{Madero University}[Puebla, M\'exico]
\begin{positions}
  \entry{Teaching Lecturer in Mechatronic Engineering}{Jan 2012~--~Jan 2013}
\end{positions}
\begin{itemize}
	\item I proposed and supervised the following students projects: Haptic Referee Glove,
	Lightmetre and Pychometre Sensors, Smart Irrigation, Persistent Of Vision Bicycle Wheel
	and a Delta Robot Structure 
	\href{https://sites.google.com/site/perezxochicaleprojects/studentprojects}{(See documents and code: \faExternalLink)}.
	\item I proposed and designed a Mechatronic Laboratory which includes: 
	(i) a benchmark for laboratories in mechatronics in M\'exico and Puebla, 
	(ii) a 3D layout design and 
	(iii) minimal requirements of hardware and software for the laboratory
	\href{https://sites.google.com/site/perezxochicaleprojects/mechatronicslaboratorydesign}{(See documents and layout: \faExternalLink)}.
\end{itemize}

\subsection{INAOE}[Puebla, M\'exico]
\begin{positions}
  \entry{Research Internship in Robotics}{Sep 2003~--~Mar 2004}
\end{positions}
\begin{itemize}
  \item  I implemented a speed control for a two-degree-of-freedom robot with microcontrollers
PIC 16F84 \& 16F877 which communicated via RS-232 to LabVIEW's Virtual Instruments.
\end{itemize}

\section{Teaching and Supervision \\ Experience}
\subsection{King's College London}[London, UK]

\begin{positions}
  \entry{Supervision}{Jan 2020~--~Present}
\end{positions}

Student: Thea Bautista \printdate{Oct 2021~--~May 2022} \\
M. Eng. in Biomedical Eng: GAN-based synthetic ultrasound imaging for fetal development \\
Co-supervisors: H. Kerdegari, L. Peralta-Pereira, and R. Aughwane

Student: Guilherme Gomes de Figueiredo \printdate{Jun 2021~--~Aug 2021} \\
Summer Project: synthetic ultrasound imagining with AI

Student: Amal Hussein \printdate{Jun 2021~--~Aug 2021} \\
Summer Project: Ultrasound-guidance simulator

Student: Alexander Mitton \printdate{Jan 2020~--~Sep 2020} \\
M.Sc. Project: Vibro-tactile stimulator for dystonia research \\
Co-supervisors: C. Bergeles, V. Mcclelland and A. Worley

Student: Alexander Mitton \printdate{Jan 2020~--~Sep 2020} \\
M.Sc. Project: Vibro-tactile stimulator for dystonia research \\
Co-supervisors: C. Bergeles, V. Mcclelland and A. Worley \\ 

\begin{positions}
  \entry{Teaching Associate}{Jan 2020~--~Present}
\end{positions}
\begin{itemize}
	\item Medical Robotics. Lecturer: Dr. Hongbin Liu \printdate{Jan 2021~--~Apr 2021}
	\item Medical Robotics. Lecturer: Dr. Christos Bergeles \printdate{Jan 2020~--~Apr 2020}
\end{itemize}
\vspace{\parskip}


\subsection{The University of Birmingham}[Birmingham, UK]
\begin{positions}
  \entry{Teaching Associate}{Aug 2014~--~Apr 2018}
\end{positions}
\begin{itemize}
	\item Engineering Maths 2. Lecturers: Prof. Martin Russell, Dr Carl Anthony \printdate{Jan 2018 -- Apr2018}
	\item Engineering Maths 2. Lecturer: Prof. Martin Russell \printdate{Aug 2017 -- Dec 2017}
	\item Computing for Engineering. Lecturer: Dr Sridhar Pammu  \printdate{Aug 2017 -- Dec 2017}
	\item Matlab Laboratories. Lecturer: Dr Edward Tarte  \printdate{Jan 2017 -- Apr 2017}
	\item Computing for Engineering. Lecturer: Dr Sridhar Pammu  \printdate{Aug 2016 -- Dec 2016}
	\item Small Embedded Systems. Lecturer: Prof. Chris Baber  \printdate{Aug 2016 -- Dec 2016} \\
\end{itemize}

 \newpage
\begin{positions}
	\entry{Supervision}{Jun 2018~--~ Dec 2018}
\end{positions}
Student: Dinghuang Zhang \\
M.Sc. Project: Tools for Human-Humanoid Collaboration \\
Co-supervisor: Chris Baber. 

\subsection{Bilingual Hight School TECMilenio University}[Puebla, M\'exico]
\begin{positions}
  \entry{Teaching Associate}{Aug 2013~--~Dec 2013}
\end{positions}
\begin{itemize}
	\item
	Information Technology \href{https://sites.google.com/site/perezxochicale/teaching/iit}{\faExternalLink}, 
	Euclidean Geometry  \href{https://sites.google.com/site/perezxochicale/teaching/euclidean-geometry}{\faExternalLink}, and 
	Microsoft Office Access \href{https://sites.google.com/site/perezxochicale/teaching/moa}{\faExternalLink}
\end{itemize}

\subsection{Universidad Madero}[Puebla, M\'exico]
\begin{positions}
  \entry{Teaching Associate in Mechatronic Eng.}{Jan 2012~--~Dec 2012}
\end{positions}
\begin{itemize}
	\item Fundamentals of Automation \href{https://sites.google.com/site/perezxochicale/digital-electronics}{\faExternalLink}, 
	Industrial Electronics \href{https://sites.google.com/site/perezxochicale/ie}{\faExternalLink},
	Research Projects \href{https://sites.google.com/site/perezxochicale/latex/thesistemplate}{\faExternalLink},
	Metrology \href{https://sites.google.com/site/perezxochicale/metrology}{\faExternalLink},
	Physics \href{http://goo.gl/fffnG}{\faExternalLink}, and 
	Computer Integrating Manufacturing, and Power Electronics
\end{itemize}

\subsection{Universidad Iberoamericana Puebla}[Puebla, M\'exico]
\begin{positions}
  \entry{Teaching Associate in Electronic Eng.}{Jan 2007~--~Dec 2011}
\end{positions}
\begin{itemize}
	\item Stochastic Processes \href{https://sites.google.com/site/perezxochicale/stochastic-processes-course}{\faExternalLink}, 
	Digital Signal Processing \href{https://sites.google.com/site/perezxochicale/digital-signal-processing-course}{\faExternalLink}, 
	and Analog Filters.
\end{itemize}

\subsection{Instituto Tecnol\'ogico Superior de Atlixco}[Puebla, M\'exico]
\begin{positions}
  \entry{Teaching Associate in Mechatronic Eng.}{Aug 2006~--~Jun 2007}
\end{positions}
\begin{itemize}
	\item (01/2007 - 06/2007) Electronics I, Numerical Methods, and Electricity and Magnetism. 
	\item (08/2006 - 12/2006) Electricity and Magnetism, and Electricity and Industrial Electronics 
\end{itemize}

\section{Publications}
\publications{publications.bib}

\section{Posters}
\publications{posters.bib}

\section{Talks}
\publications{talks.bib}


\section{Grants, Awards and Honours}
\begin{itemize}
\item King's Public Engagement grant for the project "FETUS: Finding a fETus with an Ultrasound Simulator" led by myself and in collaboration with Fang-Yu Lin and Shu Wang \href{https://cai4cai.ml/post/2021-01-07-miguelpegrant/}{\faExternalLink} \printdate{(07/01/2021 - 07/01/2022)}

\item Alexander Mitton won the Outstanding Individual Project award for his M.Sc. project, which I was the main supervisor, on designing a wearable, vibrotactile stimulation device for patients with dystonia \href{https://www.kcl.ac.uk/news/mscmres-healthcare-technologies-award-student-prizes-for-outstanding-performance-and-contributions-to-student-life}{\faExternalLink} \printdate{(15/10/2020)}

\item King's Health Partners grant for the project "Sensory system abnormalities in childhood dystonia" lead by Verity McClelland and in collaboration with Carlos Seneci \href{https://kclpure.kcl.ac.uk/portal/en/persons/miguel-angel-perez-xochicale(cca72683-31b7-496a-8aeb-181fd9d6a8f3)/projects.html}{\faExternalLink} \printdate{(14/04/2020 - 9/06/2020)}

\item My work ``Towards Healthy Ageing with Humanoid Robots'' was selected for a talk at the second forum of Mexican Talent, Innovation Match MX 2017, \href{https://github.com/mxochicale/InnovationMatchMX/tree/master/2017}{\faExternalLink} \href{https://github.com/mxochicale/InnovationMatchMX/blob/master/2017/presentation/IMMX-MA-0058.pdf}{\faFilePdfO} \href{https://www.youtube.com/watch?v=wNWzpdXdm5U}{\faYoutube} \printdate{11/01/2017}

\item I won the best poster award at the XIV Symposium of Mexican Students in the UK at University of Edinburgh \href{https://github.com/mxochicale/symposiummx/tree/master/2016}{\faExternalLink} \printdate{16-18/06/2016} 

\item My project of a low-cost robot was selected among 125 applications received from 35 countries and presented at the first international public entrepreneurship program in Mexico (MECATE 2015). \href{http://let-emprendimientopublico.mx/en/portfolio_category/mecate-primera-generacion-en/}{\faExternalLink} \href{https://www.youtube.com/watch?v=VjVGnwD422g}{\faYoutube} \printdate{20-24/07/2015}

\item Ph.D. scholarship by the Mexican National Council on Science and Technology. \printdate{11/2014-11/2018}

\item Markovito's team  won the first place at the Mexican Tournament of Robotics 2013 in the category at HOME where I presented a Human-Robot Interaction Dance Demo. \href{https://www.youtube.com/watch?v=Kw-lZam_qZI}{\faYoutube} \printdate{25-27/05/2013}

\item M.Sc. scholarship by the Mexican National Council on Science and Technology. \printdate{09/2004-09/2006}
\end{itemize}


\section{Skills}
\begin{description}
  \item[Programming] Python[2014-present],
R[2013-present], 
Robot Operating System (ROS)[2016-present],
GNU-Octave (or MatLab)[2009-present],
\LaTeX [2006-present], 
C and C++[2015-present],
Processing[2012-present], 
the shell[2010-present], 
GNU-emacs[2010-present],
vim[2016-present], 
pandoc[2017-present],
open-source enthusiast at GitHub (\href{https://github.com/mxochicale}{@mxochicale})[2015-present], 
and continuous integration and continuous delivery [2019-present]. 
  \item[Tools] 
GNU/Linux Operating System user (e.g. OpenSuse, Debian and Ubuntu)[2005-present]
Single-board computers and microcontrollers (e.g. 
NVIDIA Jetson Nano, RaspberryPi, BeagleBone, Arduino and PIC)[2010-present],
Inertial Measurement Units (e.g. calibration, collection and data analysis)[2013-present], 
Web design (e.g. Github pages, Jekyll)[2015-present], and
Graphic design (e.g. Inkscape, GIMP)[2014-present],
CAD design (e.g. Autodesk invetor, blender, FreeCAD)[2015-present], 
Artificial Neural Networks (e.g. PyTorch, and TensorFlow)[2017-present], and
3D printing (e.g., flsun, cura) [2019-present].
  \item[Languages] Spanish[Native], English[Fluent], Chinese[Beginner]
\end{description}




\section{Extra \\ Activities}
\subsection{King's College London}[London, UK]
\begin{positions}
  \entry{Outreach activities and scientific engagement}{Sep 2019~--~Present}
\end{positions}
\begin{itemize}
	\item Organising events in the Early Career Researcher Network of the BMEIS \printdate{01-01-2021 - present}.
	\item Participation in the Westminster Enterprise Week to engage students aged 14-18 to Biomedical Engineering \printdate{10-11-2021}
	\item Participant in the STEAM WEEK organised by the City Westminster Council to engage students aged 14-18 to STEAM \href{https://twitter.com/_mxochicale/status/1374407825607200769}{\faTwitter} \printdate{23-03-2021}
	\item Alexandra Lautarescu and I organised the Reproducible, Interpretable, Open, \& Transparent Science Club at St Thomas' Hospital \printdate{02-2020 -- 06-2020}
	\item For the event In2ScienceUK, I shared my scientific journey to young scientist on how they can become better scientist.  \printdate{20-08-2019}
	\item For the New Scientist Live, I showcased software that helps doctors to create 3D models of brain tumors using AI. \printdate{09-2019}
\end{itemize}

\subsection{University of Birmingham}[Birmingham, UK]
\begin{positions}
  \entry{Outreach activities and scientific engagement}{Aug 2014~--~Jun 2018}
\end{positions}
\begin{itemize}
\item  Finalist at the Three Minute Thesis Competition 2018. Video: \href{https://www.youtube.com/watch?v=07ewRYcS-0g}{\faYoutube} and 
GitHub: \href{https://github.com/mxochicale/3mt}{\faGithubAlt} \printdate{05-2018}

\item Research Poster Conference for 
(2015) \href{https://github.com/mxochicale/PhD/blob/master/posters/Research_Poster_Conference_UoB/2015/poster/poster.pdf}{\faImage}, 
(2016) \href{https://github.com/mxochicale/PhD/blob/master/posters/Research_Poster_Conference_UoB/2016/poster/poster.pdf}{\faImage}, and  
(2018) \href{https://github.com/mxochicale/PhD/blob/master/posters/Research_Poster_Conference_UoB/2018/poster/main/map479-poster-uob2018.pdf}{\faImage}.
GitHub: \href{https://github.com/mxochicale/PhD/tree/master/posters/Research_Poster_Conference_UoB}{\faGithubAlt}.

\item Demoing Human-Robot Activities at the Undergraduate Open Days. GitHub: \href{https://github.com/mxochicale/opendayuob-hridemo}{\faGithubAlt}. \printdate{2014--2018}

\item Coordinator of the Science Seminars for the Mexican Society.  GitHub: \href{https://github.com/MexicanSocietyUoB}{\faGithubAlt}, Website: \href{https://mexicansocietyuob.github.io/seminars/}{\faExternalLink}. \printdate{2017--2018}

\end{itemize}


\subsection{AIR4Children}[M\'exico \& UK]
\begin{itemize}

\item Building Artificial Intelligence and Robotics for Children (air4children) with the purpose of teaching AIR to children for free. 
Twitter: \href{https://twitter.com/air4children}{\faTwitter @air4children} 
GitHub: \href{https://github.com/air4children}{\faGithubAlt @air4chidlren} 
\printdate{2019--Present} 

\item Creation Libre Robotics, a non-profit organization aiming to freely transfer knowledge in Robotics to Mexican children.  Website: \href{https://sites.google.com/site/LibreRobotics/}{\faExternalLink}  \printdate{2013 -- 2017}

\end{itemize}

\subsection{Developer of the Website "Machine Learning for M\'exico"}[M\'exico \& UK]
\begin{itemize}
\item GitHub: \href{https://github.com/ML4MX}{\faGithubAlt}, Website: \href{https://ml4mx.github.io/website/}{\faExternalLink} 
\printdate{2013 -- 2018}
\end{itemize}

\end{document}
