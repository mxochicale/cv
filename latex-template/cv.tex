%% start of file `template.tex'.
%% Copyright 2006-2013 Xavier Danaux (xdanaux@gmail.com).
%% https://www.ctan.org/pkg/moderncv?lang=en

% This work may be distributed and/or modified under the
% conditions of the LaTeX Project Public License version 1.3c,
% available at http://www.latex-project.org/lppl/.

\documentclass[10pt,a4paper,roman]{moderncv}
% possible options include font size ('10pt', '11pt' and '12pt'), paper size ('a4paper',
% 'letterpaper', 'a5paper', 'legalpaper', 'executivepaper' and 'landscape') and font
% family ('sans' and 'roman')

% moderncv themes \moderncvstyle{casual} % style options are 'casual' (default),
% 'classic', 'oldstyle' and 'banking'
\moderncvstyle{casual}                             % style options are 'casual'
                                                    % (default), 'classic', 'oldstyle' and
                                                    % 'banking'
\moderncvcolor{black}                               % color options 'blue' (default),
                                                  % 'orange', 'green', 'red', 'purple',
                                                  % 'grey' and 'black'

%\renewcommand{\familydefault}{\sfdefault} % to set the default font; use '\sfdefault' for
%the default sans serif font, '\rmdefault' for the default roman one, or any tex font name
%\nopagenumbers{} % uncomment to suppress automatic page numbering for CVs longer than one
%page

% character encoding \usepackage[utf8]{inputenc} % if you are not using xelatex ou
%lualatex, replace by the encoding you are using

%\usepackage{CJKutf8} % if you need to use CJK to typeset your resume in Chinese, Japanese
%or Korean

% adjust the page margins
\usepackage[scale=0.82]{geometry}
%\setlength{\hintscolumnwidth}{3cm} % if you want to change the width of the column with
%the dates

%\setlength{\makecvtitlenamewidth}{10cm} % for the 'classic' style, if you want to force
%the width allocated to your name and avoid line breaks. be careful though, the length is
%normally calculated to avoid any overlap with your personal info; use this at your own
%typographical risks...

% personal data
\name{Miguel}{Xochicale}

\title{Curriculum vitae -- June 2016}    % optional, remove / comment the line if not wanted
\address{Edgbaston}{Birmingham B15 2TT}{United Kingdom}

% optional, remove / comment the line if not wanted;
% the "postcode city" and "country" arguments can be omitted or provided empty

%  \phone[mobile]{0744 }                   % optional, remove / comment the line if
                                                 % not wanted; the optional "type" of the
                                                 % phone can be "mobile" (default),
                                                 % "fixed" or "fax"
\phone[fixed]{+44~(0)~121 41 47511}

% \phone[fax]{+3~(456)~789~012}

\email{perez.xochicale@gmail.com}                               % optional, remove /
                                                                % comment the line if not
                                                                % wanted

% \email{map479@bham.ac.uk} % optional, remove / comment the line if not wanted
\homepage{mxochicale.github.io}                         % optional,
                                                                         % remove /
                                                                         % comment the
                                                                         % line if not
                                                                         % wanted
\social[github]{mxochicale}                              % optional, remove / comment the
                                                         % line if not wanted

\social[linkedin]{miguel-perez-xochicale-957074102}                        % optional,
                                                                            % remove /
                                                                            % comment the
                                                                            % line if not
                                                                            % wanted
\social[twitter]{\_mxochicale}                             % optional, remove / comment
                                                              % the line if not wanted


% \extrainfo{additional information} % optional, remove / comment the line if not wanted

\photo[84pt][0.4pt]{mxochicale_april2016}                       % optional, remove / comment the line
                                                    % if not wanted; '64pt' is the height
                                                    % the picture must be resized to,
                                                    % 0.4pt is the thickness of the frame
                                                    % around it (put it to 0pt for no
                                                    % frame) and 'picture' is the name of
                                                    % the picture file

% \quote{
% I am a hard working and enthusiastic person. Principles of responsability,
% human kidness and honesty guide me through my life decisions.
% } % optional, remove / comment the line if not wanted

% to show numerical labels in the bibliography (default is to show no labels); only useful
% if you make citations in your resume

% \makeatletter
% \renewcommand*{\bibliographyitemlabel}{\@biblabel{\arabic{enumiv}}}
% \makeatother
%\renewcommand*{\bibliographyitemlabel}{[\arabic{enumiv}]}% CONSIDER REPLACING THE ABOVE BY THIS

% bibliography with mutiple entries
% \usepackage{multibib}
% \newcites{book,misc}{{Books},{Others}}

%----------------------------------------------------------------------------------
%            content
%----------------------------------------------------------------------------------
\begin{document}


%-----       resume       ---------------------------------------------------------
\makecvtitle

\section{Research Interests}
I am interested in the field of Human Activity Recognition and Human-Robot Interaction.
As a doctoral researcher I am specifically gaining deeper understanding of the variability
of human movements using Non-linear Dynamics and Machine Learning Algorithms
to create novel analysis and interpretation of signals collected through a network of inertial measurement units.


\section{Education}

\cventry{11/2014 -- Present}{Ph.D. in Electronic
  and Computer Engineering}{University of Birmingham}{UK}{}
  {Thesis: Automatic Indentification of Movement Variability.
   \\ Advisors: Professor Chris Baber and  Professor Martin Russell  }

\cventry{09/2004 -- 09/2006}{M.Sc. in Electronics}
  {Instituto Nacional de Astrof\'isica, \'Optica y Electr\'onica (INAOE)}{M\'exico}{}
  {Thesis:Digital Filter FIR with less multipliers
  \href{https://sites.google.com/site/perezxochicale/docs/MScThesis.pdf}{\faFilePdfO}
  \\ Advisor: Gordana Jovanovic Dolecek}

\cventry{08/1999 -- 09/2004}{B.Eng. in Electronics}{Instituto Tecnol\'ogico de Puebla}{M\'exico}{}
  {Thesis: Speed control of a Robot of two degrees of freedom.
  \href{https://sites.google.com/site/perezxochicale/docs/BachelorProject.pdf}{\faFilePdfO}
  \\ Advisor: Esteban Torres Leon}


\section{Professional Experience}

\cventry{02/2013 -- 08/2013}{Research Assistant}{INAOE's Robotics Laboratory}{M\'exico}{}
{Detailed achievements: A Human-Robot Interaction Demo Dance was implemented, it was based on a
ZSTAR3 Radio Frequency single three-axis accelerometer and a Patrolbot mobile robot.
We explored four gestures wearing the accelerometer in the left writs
in order to create simple dance activities with the Patrolbot mobile robot.
For further information go to \href{https://sites.google.com/site/perezxochicale/projects/demodance}{\faExternalLink}.}

\cventry{01/2012 -- 01/2013}{Invited Lecturer}
{Universidad Madero}{Puebla, M\'exico}{}
{Detailed achievements: A design of a Mechatronic Laboratory was proposed, it includes a benchmark for
Mechatronic's laboratories in M\'exico and Puebla, a 3D layout design and minimal
requirements of hardware and software for the laboratory.
For further information go to \href{https://sites.google.com/site/perezxochicaleprojects/mechatronicslaboratorydesign}{\faExternalLink}.
}

\cventry{09/2003 -- 03/2004}{Internship}
{INAOE}{M\'exico}{}
{Detailed achievements:
Implementation of a speed control with both Microcontrollers PIC 16F84 and 16F877
via Serial Communication RS-232 using Virtual Instruments on LabVIEW.
}

\section{Teaching Experience}

\cventry{08/2013--12/2013}{Teacher}
{Bilingual Hight School at TECMilenio University}{Puebla,M\'exico}{}
{Courses:
Information Technology \href{https://sites.google.com/site/perezxochicale/teaching/iit}{\faExternalLink},
Euclidian Geometry  \href{https://sites.google.com/site/perezxochicale/teaching/euclidean-geometry}{\faExternalLink}
and
Microsoft Office Access \href{https://sites.google.com/site/perezxochicale/teaching/moa}{\faExternalLink}
}

\cventry{Spring 2012 -- Autumn 2012}{Invited Lecturer in Mechatronic Engineering}
{Universidad Madero}{Puebla, M\'exico}{}
{Courses: Fundamentals of Automation
\href{https://sites.google.com/site/perezxochicale/digital-electronics}{\faExternalLink},
Industrial Electronics \href{https://sites.google.com/site/perezxochicale/ie}{\faExternalLink},
Research Projects \href{https://sites.google.com/site/perezxochicale/latex/thesistemplate}{\faExternalLink},
Metrology \href{https://sites.google.com/site/perezxochicale/metrology}{\faExternalLink},
Physics \href{http://goo.gl/fffnG}{\faExternalLink},
Computer Integrating Manufacturing, and Power Electronics
}

\cventry{Spring 2007 -- Spring 2012}{Invited Lecturer in Electronic Engineering}
{Universidad Iberoamericana Puebla}{M\'exico}{}
{Courses: Stochastic Processes Course
\href{https://sites.google.com/site/perezxochicale/stochastic-processes-course}{\faExternalLink},
Digital Signal Processing
\href{https://sites.google.com/site/perezxochicale/digital-signal-processing-course}{\faExternalLink}
and Analog Filters.
}

\cventry{08/2006 -- 06/2007}{Invited Lecturer in Mechatronic and Electric  Engineering}
{Instituto Tecnol\'ogico Superior de Atlixco}{M\'exico}{}
{Courses: Electronics I, Numerical Methods, and Electricity and Magnetism. (January-June 2007.)
Electricity and Magnetism, and Electricity and Industrial Electronics ( August-December 2006)
}


\section{Awards and Honours}


\cvitem{16-18/06/2016}{
I won a shared first prize for presenting one of the two best posters of
the XIV Symposium of Mexican Students in the UK at University of Edinburgh.
\href{https://github.com/mxochicale/symposiummx/tree/master/2016}{\faExternalLink}.}


\cvitem{20-24/07/2015}{A low-cost robot prototype was selected amoung 125 applications received from 35 countries
and presented at an international public entrepreneurship program in Mexico (MECATE 2015).
Video of the prototype interacting with children
\href{https://www.youtube.com/watch?v=VjVGnwD422g}{\faYoutubePlay}.}

\cvitem{11/2014-11/2017}{
Full PhD Degree Scholarship in the UK
from the Mexican National Council on Science and Technology (CONACyT).}

\cvitem{25-27/05/2013}{
First place at the Mexican Tournament of Robotics 2013 in the cathegory at HOME.
I presented a Human-Robot Interation (HRI) Dance Demo \href{https://www.youtube.com/watch?v=Kw-lZam_qZI}{\faYoutubePlay}. }

\cvitem{09/2004-09/2006}{
Full Master Degree Scholarship in M\'exico from the CONACyT. }


\section{Languages}
\cvitemwithcomment{English}{IETLS Band Score 6.0: Listening 6.0, Reading 7.0, Writing 6.0, Speaking 5.5.}{11/01/14}
\cvitemwithcomment{Spanish}{Native tongue}{}


\section{Computer skills}
\cvitem{Operative Systems}{Windows(5 years), GNU-Linux Ubuntu and Debian (10 years)}
\cvitem{Programing and markup languages}{R(2 years), GNU-Octave(3 years), GNU-emacs(2 years), MATLAB(2 years), C++(1 year),
Kdevelop, LabVIEW(1 year), Arduino(3 years), Processing(1 year),
\LaTeX (5 years), and  Robot Operating System (ROS) (3 months).
For further references go to my github account: \href{https://github.com/mxochicale}{https://github.com/mxochicale}.
}


\section{Extracurricular Activities}


\cvitem{01-06/2016}{
I am developing a workshop to teach children how to build low cost robots for outreach communities at the University of Birmingham.
}

\cvitem{11/2013}{
I like to play around with the Kinect sensor and build low-costs robots
\href{https://sites.google.com/site/perezxochicale/projects/iss}{\faExternalLink}. }

\cvitem{06/2013}{
I founded LibrE Robotics, a non-profit organization, to transfer knowledge in Robotics
for children to build conditions for a better world  \href{https://sites.google.com/site/librerobotics/}{\faExternalLink}. }

\cvitem{11/2012}{
I adviced different students projects at Madero University on November 2012
: Haptic Referee Glove Lightmetre and Pychometre using Arduino, Smart Irrigation,
Persistent Of Vision Bicycle Wheel and a Delta Robot Structure.
\href{https://sites.google.com/site/perezxochicaleprojects/studentprojects}{\faExternalLink}}


%% \section{Interests}
%% \cvitem{hobby 1}{Description}
%% \cvitem{hobby 2}{Description}
%% \cvitem{hobby 3}{Description}


% \section{Extra 2}
% \cvlistdoubleitem{Item 1}{Item 4}
% \cvlistdoubleitem{Item 2}{Item 5\cite{book1}}
% \cvlistdoubleitem{Item 3}{Item 6. Like item 3 in the single column list before,
% this item is particularly long to wrap over several lines.}

% \section{References}
% \begin{cvcolumns}
%   \cvcolumn{Category 1}{\begin{itemize}\item Person 1\item Person 2\item Person 3\end{itemize}}
%   \cvcolumn{Category 2}{Amongst others:\begin{itemize}\item Person 1, and\item Person 2\end{itemize}(more upon request)}
%   \cvcolumn[0.5]{All the rest \& some more}{\textit{That} person, and \textbf{those} also (all available upon request).}
% \end{cvcolumns}


% Publications from a BibTeX file without multibib for numerical labels:
%  \renewcommand{\bibliographyitemlabel}{\@biblabel{\arabic{enumiv}}}% CONSIDER MERGING
%  WITH PREAMBLE PART to redefine the heading string ("Publications"):
%  \renewcommand{\refname}{Articles}

\nocite{*}
\bibliographystyle{plain}
\bibliography{publications}                        % 'publications' is the name of a BibTeX file

% Publications from a BibTeX file using the multibib package
%\section{Publications}
%\nocitebook{book1,book2}
%\bibliographystylebook{plain}
%\bibliographybook{publications}                   % 'publications' is the name of a BibTeX file
%\nocitemisc{misc1,misc2,misc3}
%\bibliographystylemisc{plain}
%\bibliographymisc{publications}                   % 'publications' is the name of a BibTeX file



\end{document}


%% end of file `template.tex'.
